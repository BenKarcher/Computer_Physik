\documentclass[ngerman]{scrartcl} 

\KOMAoptions{fontsize=12pt, paper=a4}
\KOMAoptions{DIV=11}
\usepackage[utf8]{inputenc}		% Direkte Eingabe von ä usw.
\usepackage[T1]{fontenc}               	% Font Kodierung für die Ausgabe
\usepackage{babel}			% Verschiedenste sprach-spezifische Extras
\usepackage[autostyle=true]{csquotes}	% Intelligente Anführungszeichen
\usepackage{amsmath}		% Mathematischer Formelsatz mit zusätzlichen mathematischen Schriften und Symbolen
\usepackage{amssymb}		% Mathematischer Formelsatz mit zusätzlichen mathematischen Schriften und Symbolen
\usepackage{physics}			% Differentialgleichungen
\usepackage{listings}			% Zum Einbinden von Programmcode verwenden wir das listings-Paket
\usepackage[dvipsnames]{xcolor}	% um Elemente von Befehlen farblich zu unterstützen
\usepackage[varg]{txfonts}             	 % Schönere Schriftart
\usepackage{graphicx}		% Paket um externe Graphiken einzufügen
\RequirePackage[backend=biber, style=numeric]{biblatex} % Literaturverzeichnis
\usepackage{hyperref} 		% um klickbare Elemente in Ihrem PDF-Ausgabedokument zu erzeugen
\RequirePackage[all]{hypcap} 		% ergänzend zu hyperref
\usepackage{siunitx}			% Intelligentes Setzten von Zahlen und Einheiten
\usepackage{enumitem}		% Aufzählungsarten
\usepackage{fancyhdr}
\usepackage{float}			% Bild genau an der gewünschten Stelle [H] einfügen, 								% \begin{figure}[H]		\includegraphics[...]{{file name}}		\end{figure} 

\setlength\parindent{0pt} 		% Sets paragraph indentation to 0

\lstset{				% Deutsche Umlaute
	basicstyle=\ttfamily,    
	literate={~} {$\sim$}{1} 	% set tilde as a literal
	{ö}{{\"o}}1
	{ä}{{\"a}}1
	{ü}{{\"u}}1
	{ß}{{\ss}}1
	{Ö}{{\"O}}1
	{Ä}{{\"A}}1
	{Ü}{{\"U}}1
}
\lstset{
	numbers=left, 				% Line numbering
	numberstyle=\footnotesize, 			% Size of numbers
	basicstyle=\ttfamily\small, 			% Style and Size of Text
	backgroundcolor=\color{White}, 		% Background Color
	language=Python, 				% Language of Code
	commentstyle=\color{Maroon}, 			% Color and Style of Comments
	stringstyle=\color{OliveGreen}, 			% Color of Strings
	showstringspaces=false,
	morekeywords={import,from,class,def,for,while,if,is,in,elif,else,not,and,or,print,break,continue,return,True,False,None,access,as,del,except,exec,finally,global,import,lambda,pass,print,raise,try,assert}, 												% Definition of new keywords that will be highlighted
	keywordstyle=\color{RoyalBlue}			% Color and Style of Keywords
}



\pagestyle{fancy}
\fancyhf{}
\rhead{Ben Karcher, Annika Hoverath}
\lhead{Computerphysik - Abgabe 6}
\rfoot{Seite \thepage}

\title{Computerphysik - Abgabe 6}
\date{\today}


\begin{document}
	% Auf 3 setzen, da es beim ersten Chapter um 1 hochgezählt wird. 3+1=4
	\setcounter{section}{9}
	\thispagestyle{fancy}
	\renewcommand{\thesection}{H.\arabic{section}:}
	\renewcommand{\thesubsection}{H\arabic{section}.\arabic{subsection}}

\section{Schnelle \textsc{FOURIER} Transformation (FFT) und Korrelationen}
\subsection{diskrete FOURIER-Transformationen}
Es gelte: 
\begin{align}
[\mathcal{F}f]_k=\sum \limits_{j=0}^{N-1}f_j \exp(2\pi ikj/N), \qquad [\mathcal{F}^{-1}f]_j=\frac{1}{N}\sum \limits_{k=0}^{N-1}f_k \exp(-2\pi ikj/N)
\end{align}
Zudem ist die diskrete Korrelation für ein Zeitgitter der Länge N definiert durch:
\begin{align}
(f \odot g)_k:=\sum\limits_{j=0}^{N-1}f_{j+k}g_j
\end{align}
Im Folgenden werden wir zeigen, dass die Relationen gelten:
\begin{align}
[\mathcal{F}(f \odot g)]_k=[\mathcal{F}f]_k[\mathcal{F}g]_{-k}=[\mathcal{F}f]_k\overline{[\mathcal{F}\overline{g}]_k}
\end{align}
%beweisen
%%%%%%%%%%%%%%%%%%%%%%%%%%%%%%%%%%%%%%%%%%%%%%%%%%%%%%%%%%%%%%%%%%%%%%%%%%%%
\section{Graphische Auswertung für große Amplituden}
Sei jetzt ein Signal gegeben durch:
\begin{align}
s(t)=\left\{\begin{matrix}sin(2\pi \alpha t) & f"ur \ 0\leq t\leq t_{max}\\ 0 & sonst \end{matrix}\right\}
\end{align}
mit einer Konstanten $\alpha$ und einer maximalen Zeit $t_{max}$, die kleiner als die Länge des Gesamtimpulses $t_{max}<<\frac{l_{Gesamtimpuls}}{c}$ ist. Nun entsteht ein gestörtes, zeitversetztes Echo mit:
\begin{align}
	e(t)=s(t-t_L)+r(t)+b \sin(2\pi \beta t)
\end{align}
mit dem Störterm $r(t)$, der ein Rauschen mit einer Amplitude $a$ beschreibt und einem sinusförmigen Nebensignal mit einer Amplitude $b$ und einer Frequenz $\beta$. 
Wir simulieren ein Rauschsignal mithilfe des Mersenne-Twisters, einem Pseudo-Zufallszahlen-Generator. Eine Funktion genrand\_res53() generiert Zufallszahlen zwischen $0$ und $1$, sodass die Verschiebung und Reskalierung geschreiben werden kann als: 
\begin{align}
	2 \cdot a \cdot (genrand\_res53()-0.5) \qquad \in [-a,a]
\end{align}
Im Folgenden stellen wir $|e(t)|^2$ graphisch für verschieden große Amplituden dar. Dabei wählen wir $a=b=0.5,1.0,2.0,4.0$ und die Frequenzen $\alpha=10$ und $\beta=1$ und für die Zeitverzögerung $t_L=50$. Dabei stellen wir fest, dass wir ab ... die Zeitverzögerung nicht mehr mit dem Auge ablesen können.
% Graphen und Wert
%%%%%%%%%%%%%%%%%%%%%%%%%%%%%%%%%%%%%%%%%%%%%%%%%%%%%%%%%%%%%%%%%%%%%%%%%%%
\section{analytische Korrelation von $e \odot s$}
Des Weiteren bestimmen wir mithilfe der FFT-Routine analytisch die Korrelation $e \odot s$:
\begin{align}
(e \odot s)_k=\biggl[ \mathcal{F}^{-1}\biggl( [\mathcal{F}e]\overline{[\mathcal{F}\overline{s}]}\biggr)\biggr]_k=\biggl[\mathcal{F}^{-1}\biggl([\mathcal{F}e]\overline{[\mathcal{F}\overline{s}]}\biggr)\biggr]_k
\end{align}
Daraufhin stellen wir $|(e \odot s)(t)|^2$ für verschieden große Störungen graphisch dar. Dabei setzen wir $N=2^m$ mit $m=13$ und prüfen den Fall, ob man $t_L$ sicher bestimmen kann. 
%%%%%%%%%%%%%%%%%%%%%%%%%%%%%%%%%%%%%%%%%%%%%%%%%%%%%%%%%%%%%%%%%%%%%%%%%%%%
\section{Vergleich der analytischen und numerischen Korrelation von $e \odot s$}
Jetzt betrachten wir die Amplituden $a=b=0$. Wir zeigen, dass die Korrelation 
\begin{align}
(e \odot s)(t):= \int_{-\infty}^{\infty}d\tau e(t+\tau)s(\tau)
\end{align}
gegeben ist durch:
\begin{align}
(e \odot s)(t) & = & 0, \qquad t<t_l-t_{max}\\
(e \odot s)(t) & = & \frac{1}{2}\cos(2\pi \alpha (t-t_l))(t_{max}-t_l+t) \\
&& -\frac{1}{4\pi \alpha} \cos(2\pi \alpha t_{max})\sin(2\pi \alpha (t_{max}-t_l+t)), \qquad t_L-t_{max}<t<t_L \\
(e \odot s)(t) & = & \frac{1}{2} \cos(2\pi \alpha(t-t_L))(t_L-t+t_{max})\\
&& -\frac{1}{4\pi \alpha}\cos(2\pi \alpha t_{max})\sin(2\pi \alpha (t_{max}+t_l-t)), \qquad t_L<t<t_l+t_{max}\\
(e \odot s)(t) & = & 0, \qquad t>t_L+t_{max}\\ 
\end{align}
Anschließend vergleichen wir unser analytisches Ergebnis mit der zuvor oben numerisch bestimmten Lösung. 
%%%%%%%%%%%%%%%%%%%%%%%%%%%%%%%%%%%%%%%%%%%%%%%%%%%%%%%%%%%%%%%%%%%%%%%%%%%%
\section{"Zwitscher"-Impuls}
Zu guter Letzt verbessern wir unser Verfahren mit einem Signal, einem sogenannten "Zwitscher"-Impuls
\begin{align}
s(t)=\left\{\begin{matrix}sin(2\pi \alpha(t) t) & f"ur \ 0\leq t\leq t_{max}\\ 0 & sonst \end{matrix}\right\}
\end{align}
mit einer linear variablen Frequenz $\alpha(t) = \alpha_0 + \alpha_1 t $.
Zuletzt untersuchen wir noch unseren Störterm aus Teilaufgabe 3 mit dieser variablen Störfrequenz unter der Wahl von $\alpha_0=5.0$ und $\alpha_1=1.0$.
\end{document}